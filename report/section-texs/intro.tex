
% Motivation+problem statement

% Prior work

% Key ideas

% Evaluation results

% Contribution summary

% \textbf{Background.}
NAT is a common network function between local-area network (LAN) and wide-area
network (WAN). 
It tracks L4 connections (i.e. TCP and UDP), translates IP addresses and ports to 
provide transparent address reuse for different clients in LAN.
Implementing NAT in software is convenient, but the massive data traffic (10Gbps or higher) will incur luxurious CPU waste and high packet latency. Therefore, it's common to offload NAT down into hardware such as switches and network interface cards (NIC). 
However, switches and NICs are usually implemented via application-specific integrated circuit (ASIC) which requires a long and expensive cycle from design to production. That makes vendors refuse to put such function into some products with a small audience, such as performant wireless NIC. 



\textbf{Motivation \& key ideas.} 
Under the circumstances above, the Field Programmable Gate Array (FPGA) comes into consideration naturally. 
It allows users to implement custom hardware logic including NAT. Therefore, in this project, we explore the possibility of implementing NAT in FPGA and show the performance gain from hardware-oriented design. 

\textbf{Prior works.} We survey prior works including other hardware offloading approaches and advanced NAT designs. Programmable switches with P4 language support line-rate match-action processing by which NAT can be implemented naturally. 
\footnote{https://github.com/p4lang/switch/blob/master/p4src/nat.p4}
Some high-performance NICs such as Nvidia (Mellanox) ConnectX-4/5/6 can support NAT offloading itself with proper configuration. However, they are all wired NICs for the data center network.
There are also already some trials to implement NAT on FPGA such as the NetFPGA platform. More details can be found at the footnote link.
\footnote{https://www.cl.cam.ac.uk/~osc22/docs/edv10\_nat\_fpga.pdf}

\textbf{Evaluation Results.} In our evaluation, FPGA-based NAT shows a 7.7\% performance gain in throughput than software-based NAT and better scalability over connection numbers.

\subsection{Contribution Summary}
\begin{itemize}
    \item Exploration and System Design: Yongtong Wu
    \item Hardware Impl. and Testing: Rilin Huang, Yongtong Wu
    \item Software and Testbed Configuration: Jialiang Zhang
    \item Evaluation \& Live Demo Video: Yongtong Wu
    \item Poster: Jialiang Zhang
    \item Report Writing: Rilin Huang, Yongtong Wu
\end{itemize}