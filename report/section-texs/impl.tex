

We show our implementation in this section, including the testbed setting, the whole workflow, and implementation results such as code lines and FPGA resource consumption. We implement our design based on Xilinx's official example of the Ethernet 10/25G system.
\subsection{Testbed Configuration}
We implemented our NAT with AMD (Xilinx) Zynq UltraScale+ MPSoC ZCU102 (called the board later). The evaluation setup includes the board, a router, and a desktop machine.
The board runs Linux 4.19.0 on its four PS ARM cores. 
The desktop runs Ubuntu 18.04 equipped with Intel i7-9700K and an NVIDIA  ConnectX-3 (Mellanox) network interface card, 
whose line rate reaches 10Gbps, with \verb|mlx4| driver version 4.14.142-ubwins+. The MTU is configured as 9000B to get full performance.
The performance of the router is not critical since we only use it to ssh connect to the board.





\textbf{Network Topology.} 
The board's 1Gbps interface and the on-board NIC of the desktop are both connected to the router and are configured under the same subnet. 
For the data plane, we connect the 10Gbps interface of the board and an interface of ConnectX-3 back-to-back to fully investigate the potential of our FPGA NAT.

\subsection{Implementation Workflow}


\begin{figure*}[h]
    \centering
    \includegraphics[width=500pt]{images/implworkflow.pdf}
    \caption{Our implementation workflow. }
    \label{fig:workflow}
    \Description{111}
\end{figure*}

We elaborate on our detailed implementation workflow, which includes Verilog design and testing, Vivado block design, Petalinux image building, testbed deployment, and Vitis cross-compile in this subsection. We use Vivado 19.02, Vitis 19.02, and Petalinux 19.02. Figure \ref{fig:workflow} shows our workflow.

\textbf{Hardware implementation and testing.} We use Verilog and Verilator to implement and test our design in about 300 lines of Verilog code and 1.5k lines of C++ code. 


\begin{table}
    \caption{ZCU102 Resource Consumption}
    \label{tab:res}
    \begin{tabular}{c|cccc}
      \toprule
        &LUT&FF&BRAM&DSP\\
      \midrule
      Origin & 18923 & 18315 & 139.5 & 0\\
      Ours (table size 1024) & 33373 & 47067 & 142.0 & 0\\

      \midrule
      Ours \slash Total Resource (\%) & 12.18 & 8.59 & 15.57 & 0\\

    \bottomrule
  \end{tabular}
\end{table}


\begin{figure*}[t]
    \centering
    \includegraphics[width=400pt]{images/brd.png}
    \caption{The key part of our board design. }
    \label{fig:brd}
    \Description{}
\end{figure*}


\begin{figure}[t]
    \centering
    \includegraphics[width=\linewidth]{images/brdimpl.png}
    \caption{The post-synthesis netlist with NAT part highlighted with yellow.}
    \label{fig:brdimpl}
    \Description{}
\end{figure}

\textbf{Vivado block design and synthesis.} We use Vivado to insert our Verilog implementation as custom IPs, and then synthesize, implement, and generate bitstream. Finally, we export the hardware to an \verb|.xsa| file. The key part of our board design is shown in Figure \ref{fig:brd} and the netlist is shown in Figure \ref{fig:brdimpl}.
The resource consumption is shown in Table \ref{tab:res}.
The whole design (with table size 1024, original 10/25G Ethernet Subsystem included) consumes 12.18\% LUTs, 0.58\% LUTRAMs, 8.59\% FFs, 15.57\% BRAMs.
The delta of resource consumption between the original example and our modified design is 14464 LUTs, 13252 FF, 2.5 BRAM, and 0 DSP.
% Note that ConnectX-3 has the feature of L2, L3, and L4 checksum offloading. 
% We cannot disable it directly due to driver bugs.
Note that we use a custom Ethernet protocol number rather than \verb|0x0800| to prevent potential complexities such as NIC checksum-offloading and Linux kernel sanity check. Therefore, we do not need to insert CSU IP.

\textbf{Petalinux image building.} Petalinux is a tool to build tiny embedded Linux images.
The built image can be placed into an SD card to boot up the whole board. 
We build the images following the official tutorial in a Ubuntu 18.04 docker container over AMD Ryzen 7 7700 (4.8 Ghz full core). It takes up to several hours.

\textbf{Vitis cross-compilation.} We need to cross-compile the C++ source code since the CPUs on the board are ARM cores. Vitis provides cross-compile support for different Zynq platforms. The generated ELF file can be sent to the board via \verb|scp| directly.





% ○FPGA型号
% ○网络拓扑
% ○workflow,以及所有工具的版本
% ○soarx以及fpga上驱动版本号
% ○soarx、fpga上的Linux发行版以及内核版本
% ○网卡型号光纤线?光模组
% ○FPGA综合后占的资源
% ○代码行数
% ○
% ○表大小1024,MTU9000
% ○因为用了不同协议号,所以不需要插入CSU IP
% ○Detailed setting

% The whole design (with table size 1024, original 10/25G Ethernet Subsystem included) consumes 12.18\% LUTs, 0.58\% LUTRAMs, 8.59\% FFs, 15.57\% BRAMs.
% The delta of resource consumption between the original example and our modified design is 14464 LUTs, 13252 FF, 2.5 BRAM, and 0 DSP.