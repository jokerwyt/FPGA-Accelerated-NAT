% ○discussion & future work:
% ■更好的实现方式:
% ●完全不经过PS
% ●more advanced NAT designs?
% ●
% ■更好的实验设计:
% ●FPGA性能弱对实验的影响
% ●软件NAT的实现方式对性能的影响
% ●实验不够详尽?还有什么关键的实验:

In this section, we will discuss some possible improvement in our system design and evaluation.

\textbf{Better evaluation method: bypassing PS.} Our evaluation result shows that the weak performance of ZCU102 hardware prevents us from fully investigate the potential of our hardware design. Thus, bypassing the PS part of the board may be a better choice to get precise evaluation result. To be specific, the received Ethernet packets should being send back to another SFP interface after they are processed by our custom IPs in the PL logic, instead of passing them to those weak ARM CPUs.

\textbf{More efficient NAT design.} The NAT we implemented use little resource at a cost of wasting more cycles. In fact, we can utilize the hardware parallelism by introducing structure like full-associated cache in our NAT table which promise one cycle latency.

\textbf{Lack of connection state management.} A functional NAT should track connection states to release the NAT table entries in time. Due to time limit, we just discuss and come up with a possible design and did not implement this part. 